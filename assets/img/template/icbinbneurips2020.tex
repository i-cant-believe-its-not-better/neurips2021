\documentclass{article}

% if you need to pass options to natbib, use, e.g.:
%     \PassOptionsToPackage{numbers, compress}{natbib}
% before loading neurips_2020

% ready for submission
% \usepackage{neurips_2020}

% to compile a preprint version, e.g., for submission to arXiv, add add the
% [preprint] option:
%     \usepackage[preprint]{neurips_2020}

% to compile a camera-ready version, add the [final] option, e.g.:
%     \usepackage[final]{neurips_2020}

% to avoid loading the natbib package, add option nonatbib:
     \usepackage[nonatbib]{icbinb_neurips_2020}

\usepackage[utf8]{inputenc} % allow utf-8 input
\usepackage[T1]{fontenc}    % use 8-bit T1 fonts
\usepackage{hyperref}       % hyperlinks
\usepackage{url}            % simple URL typesetting
\usepackage{booktabs}       % professional-quality tables
\usepackage{amsfonts}       % blackboard math symbols
\usepackage{nicefrac}       % compact symbols for 1/2, etc.
\usepackage{microtype}      % microtypography

\title{Formatting Instructions For ICBINB@NeurIPS 2020}

% The \author macro works with any number of authors. There are two commands
% used to separate the names and addresses of multiple authors: \And and \AND.
%
% Using \And between authors leaves it to LaTeX to determine where to break the
% lines. Using \AND forces a line break at that point. So, if LaTeX puts 3 of 4
% authors names on the first line, and the last on the second line, try using
% \AND instead of \And before the third author name.

\author{%
  David S.~Hippocampus\thanks{Use footnote for providing further information
    about author (webpage, alternative address)---\emph{not} for acknowledging
    funding agencies.} \\
  Department of Computer Science\\
  Cranberry-Lemon University\\
  Pittsburgh, PA 15213 \\
  \texttt{hippo@cs.cranberry-lemon.edu} \\
  % examples of more authors
  % \And
  % Coauthor \\
  % Affiliation \\
  % Address \\
  % \texttt{email} \\
  % \AND
  % Coauthor \\
  % Affiliation \\
  % Address \\
  % \texttt{email} \\
  % \And
  % Coauthor \\
  % Affiliation \\
  % Address \\
  % \texttt{email} \\
  % \And
  % Coauthor \\
  % Affiliation \\
  % Address \\
  % \texttt{email} \\
}

\begin{document}

\maketitle

\begin{abstract}
  The abstract paragraph should be indented \nicefrac{1}{2}~inch (3~picas) on
  both the left- and right-hand margins. Use 10~point type, with a vertical
  spacing (leading) of 11~points.  The word \textbf{Abstract} must be centered,
  bold, and in point size 12. Two line spaces precede the abstract. The abstract
  must be limited to one paragraph.
\end{abstract}

\section{Submission of papers to ICBINB@NeurIPS 2020}

ICBINB@NeurIPS requires electronic submissions.  The electronic submission site is
\begin{center}
  \url{https://openreview.net/group?id=NeurIPS.cc/2020/Workshop/ICBINB/}
\end{center}

Please read the instructions below carefully and follow them faithfully.

\section{Style}

Papers to be submitted to ICBINB@NeurIPS 2020 must be prepared according to the
instructions presented here, in agreement with the formatting instructions from NeurIPS. Papers may only be up to eight pages long,
including figures. Additional pages \emph{containing only a section on the broader impact, acknowledgments and/or cited references} are allowed. Papers that exceed eight pages of content will not be reviewed, or in any other way considered for
presentation at the conference.

The margins in 2020 are the same as those in 2007, which allow for $\sim$$15\%$
more words in the paper compared to earlier years.

Authors are required to use the ICBINB@NeurIPS \LaTeX{} style files obtainable at the
ICBINB@NeurIPS website:
\begin{center}
	\url{https://i-cant-believe-its-not-better.github.io/}
\end{center}

\end{document}
